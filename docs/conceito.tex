\documentclass{article}
\usepackage[utf8]{inputenc}
\usepackage[brazilian]{babel}
\usepackage{hyperref}
\usepackage{xcolor}
\usepackage{enumitem}

\hypersetup{
    colorlinks=true,
    linkcolor=blue,
    urlcolor=blue,
    citecolor=blue
}

\title{\textbf{MyPolTracker: Nome a definir} \\ \large Conceito e Arquitetura}
\author{O especialista?!}
\date{10 de fevereiro de 2026}

\begin{document}
\maketitle
\tableofcontents
\newpage

\section{Dados Oficiais no Brasil}
No Brasil há \textbf{dados oficiais, públicos e com API} mostrando \textbf{quem votou em quê}. Tudo mantido pelo próprio governo.

\subsection{Câmara dos Deputados}
\subsubsection{Portal e API}
\begin{description}[style=nextline]
    \item[Portal oficial] \url{https://www.camara.leg.br}
    \item[API REST (aberta)] \url{https://dadosabertos.camara.leg.br/api/v2/}
    \item[Formato] JSON
\end{description}

\subsubsection{Recursos Disponíveis}
\begin{itemize}[leftmargin=*]
    \item Votações nominais com voto individual de cada deputado
    \item Opções de voto: Sim, Não, Abstenção, Obstrução, Ausente
    \item Projetos, sessões, proposições, partidos e legislaturas
\end{itemize}

\subsubsection{Principais Endpoints}
\begin{description}[font=\ttfamily]
    \item[/votacoes] Lista todas as votações
    \item[/votacoes/\{id\}/votos] Votos individuais de uma votação específica
    \item[/proposicoes/\{id\}] Detalhes de uma proposição
    \item[/deputados/\{id\}] Informações sobre um deputado
\end{description}

\subsection{Senado Federal}
\subsubsection{Portal e API}
\begin{description}[style=nextline]
    \item[Portal oficial] \url{https://www25.senado.leg.br}
    \item[API (dados abertos)] \url{https://www12.senado.leg.br/dados-abertos}
    \item[Catalogo de APIs] \url{https://www12.senado.leg.br/dados-abertos/dados-abertos/catalogo-de-dados-abertos}
    \item[Formatos] JSON ou XML
\end{description}

\subsubsection{Recursos Disponíveis}
\begin{itemize}[leftmargin=*]
    \item Votações no plenário com voto individual
    \item Matérias, sessões e autores
\end{itemize}

\section{Funcionalidades Propostas}

\subsection{Seguimento de Políticos}
\paragraph{Quem seguir?}
Deputados federais, senadores, governadores, prefeitos, vereadores

\paragraph{O que acompanhar?}
\begin{itemize}[leftmargin=*]
    \item Votos nominais em tempo real
    \item Alertas de novas votações
    \item Mudanças de posicionamento
\end{itemize}

\subsection{Dados e Métricas}
\begin{enumerate}[leftmargin=*]
    \item \textbf{Votações}
    \begin{itemize}
        \item Projetos de lei, PECs, MPVs
        \item Filtros por tema: saúde, educação, segurança, meio ambiente, economia
        \item Filtros por tipo de sessão: ordinárias, extraordinárias
    \end{itemize}
    
    \item \textbf{Transparência}
    \begin{itemize}
        \item Salários e benefícios
        \item Gastos públicos
    \end{itemize}
\end{enumerate}

\subsection{Visualizações}
\begin{description}[leftmargin=*]
    \item[Ementas] Resumo por tema/projeto com ementa e status
    \item[Filtros partidários] Votações e votantes por partido e bancada
    \item[Linha do tempo] Visualização cronológica de votações por período e tema
    \item[Comparações] Análise de convergência entre políticos
\end{description}

\section{Arquitetura Técnica}

\subsection{Backend}
\subsubsection{Node.js + Express}
\begin{itemize}[leftmargin=*]
    \item Agendamento de tarefas para atualização periódica de dados
    \item API REST para servir dados ao frontend
    \item Persistência com Prisma + PostgreSQL
\end{itemize}

\subsubsection{Python (Microserviço, acho meio complexo)}
\begin{itemize}[leftmargin=*]
    \item Scraping e consumo de APIs externas
    \item ETL (Extract, Transform, Load)
    \item Processamento e limpeza de dados
\end{itemize}
\subsection{Chamadas de API}
\subsubsection{Como?}
Tem q ter em mente os limites a chamadas de api, se expandir, vai ter mta gente consumindo api q n é nossa.
Talvez consumir as apis externas e popular um banco nosso? 
\subsection{Perguntas}
\begin{itemize}[leftmargin=*]
    \item Separação entre coleta de dados (Python) e API de serviço (Node.js)
    \item Cache para reduzir chamadas às APIs governamentais
    \item Processamento assíncrono para grandes volumes
\end{itemize}

\end{document}